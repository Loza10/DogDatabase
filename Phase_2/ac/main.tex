\documentclass{article}
\usepackage{graphicx}
\usepackage{fancyhdr}
\usepackage{hyperref}
\usepackage{float}
\usepackage{fancyvrb}
\usepackage{upquote}
\hypersetup{
  colorlinks   = true,
  urlcolor     = blue,   
  linkcolor    = blue,     
}

%page setup- title and author
\title{Project Phase 2}
\author{Team 025}
\pagestyle{fancy}

%header and footer
\fancyhf{}
\fancyhfoffset[L]{1cm}
\fancyhfoffset[R]{1cm}
\lhead{Phase 2 Report}
\chead{CS 6400 - Spring 2025}
\rhead{Team 025}
\lfoot{\hyperlink{toc}{Table of Contents}}
\rfoot{\thepage}

%table of contents
\begin{document}
\maketitle
\tableofcontents
\newpage

\newpage
\section{Abstract Code + SQL}

\subsection{Login}
\begin{itemize}
    \item User enters \textit{emailAddress}, \textit{password} input fields.
    \item When User clicks \textbf{Enter} button
    \begin{itemize}


\begin{Verbatim} [frame=single]
SELECT v.email 
FROM Volunteer v 
JOIN User u ON v.email = u.email 
WHERE v.email = @emailAddress
    AND u.password = @password;
\end{Verbatim}
        
        \item Validate \textit{emailAddress} and \textit{password} are not blank.
        \item If \textit{emailAddress} is found, validate \textit{password}.
        \item If validation is true, then (table from query is not empty):
        \begin{itemize}
            \item Store User object as instance that can be accessed across the application
        \end{itemize}
        \begin{itemize}
            \item Check if user is Director and save the result in User object.
            \begin{Verbatim} [frame=single]
SELECT 1 FROM Director dir 
WHERE dir.email = @emailAddress;
            \end{Verbatim}
            \item Navigate to \textbf{\textit{Dog Dashboard}} form.
        \end{itemize}
        \item Else:
        \begin{itemize}
            \item Display error message on \textbf{\textit{Login}} form.
        \end{itemize}
    \end{itemize}
\end{itemize}

\subsection{Add Dog}
\begin{itemize}
    \item Display form with inputs if \textbf{currentNumberOfDogs+1} is less than or equal to \textbf{maxCapacity}:
        \begin{itemize}
        \begin{Verbatim} [frame=single]
INSERT INTO Dog (email,name,description,age,sex,altered,
    surrender_date,surrender_phone,by_animal_control) 
VALUES (@email, @dogName, @description, @age, @sex, 
    @altered, @surrenderDate, @surrenderPhone, 
    @byAnimalControl);
        \end{Verbatim}
    \begin{Verbatim} [frame=single]
        SELECT LAST_INSERT_ID();
    \end{Verbatim}
    \item Store result into dogID variable
    \item For each breed insert into DogBreed table:
        \begin{Verbatim} [frame=single]
INSERT INTO DogBreed VALUES (@dogID, @breedName);
        \end{Verbatim}
        \begin{Verbatim} [frame=single]
INSERT INTO DogMicrochip Values (@dogID, @microchipID);
        \end{Verbatim}
        \item \textbf{dogName}: text box
        \item \textbf{breeds}: multi-select dropdown
        \begin{itemize}
            \item Multiple selections allowed
            \item If ``Unknown'' OR ``Mixed'' selected:
            \begin{itemize}
                \item Disable other selections
            \end{itemize}
        \end{itemize}
        \item \textbf{sex}: radio button
        \begin{itemize}
            \item ``Male''
            \item ``Female''
            \item ``Unknown''
        \end{itemize}
        \item \textbf{altered}: radio button
        \begin{itemize}
            \item ``Yes''
            \item ``No''
        \end{itemize}
        \item \textbf{age}: two input boxes
        \begin{itemize}
            \item \textbf{year}: converted into months (default value: 0)
            \item \textbf{months}
        \end{itemize}
        \item \textbf{description}: text box input (optional)
        \item \textbf{manufacturer}: dropdown (nullable)
        \item \textbf{microChipID}: dropdown
        \begin{itemize}
            \item Disabled until a manufacturer is chosen (default value: null)
            \item Dropdown values update once a manufacturer is selected
            \item Nullable, must be unique per dog
        \end{itemize}
        \item \textbf{surrenderDate}: calendar input
        \item \textbf{surrenderPhone}: number input
        \begin{itemize}
            \item If \textbf{byAnimalControl} is true, input is required
            \item Else, optional
        \end{itemize}
        \item \textbf{byAnimalControl}: radio button
        \begin{itemize}
            \item ``Yes''
            \item ``No''
        \end{itemize}
    \end{itemize}
    \item Verify if dog is bulldog breed and named ``Uga''
    \begin{itemize}
        \item If true - display error: name not allowed
    \end{itemize}
    \item Upon:
    \begin{itemize}
        \item Click \textbf{Submit} Button: display options \textbf{Dog Details} or \textbf{Return to Dashboard} links
        \begin{itemize}
            \item Click \textbf{Dog Details} $\rightarrow$ Jump to the \textbf{Dog Detail} task
            \item Click \textbf{Return to Dashboard} $\rightarrow$ Jump to the \textbf{Dog Dashboard} task
        \end{itemize}
    \end{itemize}
\end{itemize}

\subsection{View Dog}
\begin{itemize}
    \item Display dog profile with:
    \item dogID is retrieved from angular route url
    \begin{itemize}
\begin{Verbatim} [frame=single]
SELECT
    d.dogID,
    d.name,
    COALESCE(GROUP_CONCAT(DISTINCT b.name 
    	ORDER BY b.name SEPARATOR '/'), 'Unknown') 
        AS breeds,
    d.sex,
    d.altered,
    FLOOR(d.age / 12) AS age_years,
    d.age % 12 AS age_months,
    d.description,
    dm.microchipID,
    m.manufacturer AS microchipVendor,
    d.surrender_date,
    d.surrender_phone,
    d.by_animal_control
FROM Dog d
LEFT JOIN DogBreed db ON d.dogID = db.dogID
LEFT JOIN Breed b ON db.name = b.name
LEFT JOIN DogMicrochip dm ON d.dogID = dm.dogID
LEFT JOIN Microchip m ON dm.microchipID = m.microchipID
WHERE d.dogID = @dogID;
\end{Verbatim}
\footnote{GROUP\_CONCAT() is a function in MySQL. See \url{https://www.geeksforgeeks.org/mysql-group_concat-function/}}
        \item \textbf{dogName}
        \item \textbf{breeds} (forward slash-separated in alphabetical if multiple)
        \item \textbf{sex}
        \item \textbf{altered} status
        \item \textbf{age} (displayed in years and months)
        \item \textbf{description} (if available)
        \item \textbf{manufacturer} (if applicable)
        \item \textbf{microChipID} (if applicable)
        \item \textbf{surrenderDate}
        \item \textbf{surrenderPhone} (if applicable)
        \item \textbf{byAnimalControl} status
        \item Display expenses section:
        \begin{itemize}
            \begin{Verbatim} [frame=single]
SELECT e.category, SUM(e.amount) AS totalExpense 
FROM Expense e 
WHERE e.dogID = @dogID 
GROUP BY e.category 
ORDER BY totalExpense DESC;
            \end{Verbatim}
            \begin{Verbatim} [frame=single]
SELECT SUM(e.amount) AS grandTotal 
FROM Expense e 
WHERE e.dogID = @dogID;
            \end{Verbatim}
            \item \textbf{totalExpense} per vendor
            \item Add all \textbf{totalExpense} and display as \textbf{grandTotal}
        \end{itemize}
        \item If user.\textbf{age} is greater than 18:
        \begin{itemize}
            \begin{Verbatim} [frame=single]
SELECT v.email 
FROM Volunteer v 
WHERE TIMESTAMPDIFF(YEAR, v.birth_date, CURDATE()) 
>= 18 AND v.email = @user.emailAddress;
            \end{Verbatim}
            \footnote{TIMESTAMPDIFF() is a function in MySQL See \url{https://www.geeksforgeeks.org/timestampdiff-function-in-mysql/}}
             \footnote{CURDATE() is a function in MySQL See \url{https://www.geeksforgeeks.org/mysql-group_concat-function/}}
            \item Display \textbf{Enter New Expense} link
        \end{itemize}
        \item If \textbf{user} is Director and dog.\textbf{microChipID} is not NULL and dog.\textbf{altered} is true
        \begin{itemize}
        \begin{Verbatim} [frame=single]
SELECT d.dogID, d.name, d.altered, dm.microchipID 
FROM Dog d
LEFT JOIN DogMicrochip dm ON d.dogID = dm.dogID
WHERE d.dogID = @dogID
AND d.altered = TRUE 
AND dm.microchipID IS NOT NULL; 
            \end{Verbatim}
            \item Display \textbf{Add Adoption} button
        \end{itemize}
    \end{itemize}
\end{itemize}


\subsection{Update Dog Information}
\begin{itemize}
    \item Upon:
    \begin{itemize}
        \item Click \textbf{Enter New Expense link} then jump to \textbf{Expenses} task
        \item Click \textbf{Add Application} button then jump to \textbf{Add Application} task
    \end{itemize}
    \item If a dog is adopted, disable everything. Check if user is at least 18 for options
    \begin{Verbatim} [frame=single]
SELECT a.dogID FROM Adoption a WHERE a.dogID = @dogID;
    \end{Verbatim}
    \begin{Verbatim} [frame=single]
SELECT v.email 
FROM Volunteer v 
WHERE TIMESTAMPDIFF(YEAR, v.birth_date, 
    CURDATE()) >= 18 AND v.email = @user.emailAddress;
    \end{Verbatim}
    \footnote{CURDATE() is a function in MySQL See \url{https://www.geeksforgeeks.org/mysql-group_concat-function/}}
    \begin{itemize}
        \item \textbf{dogName} (Disabled)
        \item \textbf{breeds} (If unknown or mixed, editable. Otherwise, disable)
        \begin{Verbatim} [frame=single]
DELETE FROM DogBreed
WHERE dogID = @dogID AND name IN ('Unknown', 'Mixed');
        \end{Verbatim}
        \item For each breed insert into DogBreed table:
        \begin{Verbatim} [frame=single]
INSERT INTO DogBreed (dogID, name)
VALUES (@dogID, @breed);
        \end{Verbatim}
        \item \textbf{sex} (If unknown, editable. Otherwise, disable)
        \begin{Verbatim} [frame=single]
UPDATE Dog d
SET sex = @sex
WHERE sex = 'UNKNOWN' AND d.dogID = @dogID;
        \end{Verbatim}
        \item \textbf{altered} status (If false, editable. Otherwise, disable)
        \begin{Verbatim} [frame=single]
UPDATE Dog d
SET altered = @altered
WHERE altered = FALSE AND d.dogID = @dogID;
        \end{Verbatim}
        \item \textbf{age} (disabled)
        \item \textbf{description} (disabled)
        \item \textbf{manufacturer} (If null and user.age greater than 18, editable. Update microChipID dropdown.)
        \item \textbf{microChipID} (If manufacturer is not null and user.age is greater than 18, editable.)
        \begin{Verbatim} [frame=single]
INSERT INTO Microchip (microchipID, manufacturer)
VALUES (@microChipID, @manufacturer);
        \end{Verbatim}
        \begin{Verbatim} [frame=single]
INSERT INTO DogMicrochip (dogID, microchipID)
VALUES (@dogID, @microChipID);
        \end{Verbatim}
        \item \textbf{surrenderDate} (disabled)
        \item \textbf{surrenderPhone} (disabled)
        \item \textbf{byAnimalControl} status (disabled)
    \end{itemize}
\end{itemize}

\subsection{Add Expenses}
\begin{Verbatim} [frame=single]
INSERT INTO Expense (dogID, vendor, date, amount, category)
SELECT @dogID, @vendorName, @expenseDate, @amount, 
    @expenseCategory
WHERE (@expenseDate > (SELECT d.surrender_date
FROM Expense e
LEFT JOIN Dog d ON e.DogID = d.DogID
LEFT JOIN Adoption a ON e.DogID = a.DogID
WHERE e.DogId = @dogID
LIMIT 1 )) AND
(SELECT COALESCE(a.decision_date)
FROM Expense e
LEFT JOIN Dog d ON e.DogID = d.DogID
LEFT JOIN Adoption a ON e.DogID = a.DogID
WHERE e.DogId = @dogID LIMIT 1 ) IS NULL;
\end{Verbatim}
\begin{itemize}
    \item Display form with inputs:
    \begin{itemize}
        \item \textbf{expenseDate}: calendar input
        \item \textbf{vendorName}: text box
        \item \textbf{amount}: double
        \item \textbf{expenseCategory}: dropdown
    \end{itemize}
    \item Submission conditions:
    \begin{itemize}
        \item If dog has not been adopted and dog is surrendered:
        \begin{itemize}
            \item Allow submit
        \end{itemize}
        \item If (\textbf{expenseCategory} exists and \textbf{currentDate} is true) OR user.\textbf{age} is less than 18:
        \begin{itemize}
            \item Do not allow submit
        \end{itemize}
    \end{itemize}
\end{itemize}

\subsection{Adoption Review}
\begin{itemize}
    \item Display pending applications and applicant contact information.
    \item Display table containing:
    \begin{itemize}
        \item Applications in \textbf{Pending} state.
        \item Applicant contact details:
        \begin{itemize}
        \begin{Verbatim} [frame=single]
SELECT 
a.email, 
d.first_name, 
d.last_name, 
d.street, 
d.city, 
d.state, 
d.zip, 
d.phone 
FROM Application a
LEFT JOIN Adopter d ON a.email = d.email;
        \end{Verbatim}
            \item \textbf{firstName}
            \item \textbf{lastName}
            \item \textbf{Address}:
            \begin{itemize}
                \item \textbf{street}
                \item \textbf{city}
                \item \textbf{state}
                \item \textbf{zipCode}
            \end{itemize}
            \item \textbf{phoneNumber}
            \item \textbf{emailAddress}
        \end{itemize}
        \item Buttons for:
        \begin{itemize}
            \item \textbf{Approve}
            \begin{Verbatim} [frame=single]
INSERT INTO Adoption 
(email, date, fee, 
    is_fee_waived, decision_date, dogID)
SELECT 
    a.email, 
    a.date, 
    CASE WHEN d.by_animal_control = TRUE 
        THEN 
        ROUND(COALESCE(e.total_expense * 0.10, 0), 2) 
        ELSE 
        ROUND(COALESCE(e.total_expense * 1.25, 0), 2) 
    END AS fee, 
    CASE 
        WHEN d.name = 'Sideways' 
            AND EXISTS (
                SELECT 1 FROM DogBreed db 
                JOIN Breed b ON db.name = b.name 
                WHERE db.dogID = d.dogID 
                AND b.name LIKE '%Terrier%'
            ) 
        THEN TRUE 
        ELSE FALSE 
    END AS is_fee_waived, 
    @decDate, 
    @dogID
FROM Application a
JOIN Dog d ON d.dogID = @dogID
LEFT JOIN (
    SELECT d.dogID, SUM(e.amount) AS total_expense
    FROM Expense e
    JOIN Dog d ON e.dogID = d.dogID
    GROUP BY d.dogID
) e ON e.dogID = @dogID
WHERE a.email = @user.emailAddress AND a.date = @date;
            \end{Verbatim}
            \begin{Verbatim} [frame=single]
DELETE FROM Application 
WHERE email = @user.emailAddress 
AND date = @date;
            \end{Verbatim}
            \item \textbf{Reject}
            \begin{Verbatim} [frame=single]
INSERT INTO Rejection (email, date)
SELECT email, a.date
FROM Application a
WHERE email = @user.emailAddress AND date = @date;
            \end{Verbatim}
            \begin{Verbatim} [frame=single]
DELETE FROM Application 
WHERE email = @user.emailAddress 
AND date = @date;
            \end{Verbatim}
        \end{itemize}
    \end{itemize}
\end{itemize}


\subsection{Add Application}
\begin{Verbatim} [frame=single]
INSERT INTO Adopter
VALUES (@emailAddress, @firstName, @lastName, @phoneNumber, 
    @street, @city, @state, @zipCode, @householdSize);
\end{Verbatim}
\begin{Verbatim} [frame=single]
INSERT INTO Application
VALUES (@emailAddress, @applicationDate);
\end{Verbatim}
\begin{itemize}
    \item Display form with inputs:
    \begin{itemize}
        \item \textbf{firstName}: text box
        \item \textbf{lastName}: text box
        \item \textbf{street}: text box
        \item \textbf{city}: text box
        \item \textbf{state}: text box
        \item \textbf{zipCode}: number input (numbers only)
        \item \textbf{phoneNumber}: number input (numbers only)
        \item \textbf{emailAddress}: text box
        \item \textbf{householdSize}: number input (numbers only)
    \end{itemize}
    \item \textbf{applicationStatus} will be set to "Pending" upon submission.
    \item Submission conditions:
    \begin{itemize}
        \item If \textbf{emailAddress} and \textbf{applicationDate} exist:
        \begin{itemize}
            \item Display error message indicating only one application per day is allowed.
        \end{itemize}
    \end{itemize}
\end{itemize}


\subsection{Search Applications}
\begin{itemize}
    \item Display search input box:
    \begin{itemize}
        \item Look up via applicant's 	\textbf{lastName} (case insensitive)
    \end{itemize}
    \item Display list of applicants with:
    \begin{itemize}
    \begin{Verbatim} [frame=single]
SELECT ad.first_name, 
       ad.last_name, 
       ad.street, 
       ad.city, 
       ad.state, 
       ad.zip AS zip_code, 
       ad.phone AS phone_number, 
       ad.email AS email_address, 
       ad.household 
FROM Adopter ad 
WHERE ad.last_name LIKE '%@lastName%';
        \end{Verbatim}
        \item 	\textbf{firstName} + 	\textbf{lastName} (clickable link)
        \item Address: 	\textbf{street}, 	\textbf{city}, 	\textbf{state}, 	\textbf{zipCode}
        \item 	\textbf{phoneNumber}
        \item 	\textbf{emailAddress}
        \item 	\textbf{householdSize}
    \end{itemize}
    \item Upon selecting an applicant (via 	\textbf{firstName} + 	\textbf{lastName} link):
    \begin{itemize}
        \item Display most recent application
        \begin{Verbatim} [frame=single]
SELECT a.email, 
       a.first_name, 
       a.last_name, 
       a.phone, 
       a.street, 
       a.city, 
       a.state, 
       a.zip,
       a2.date,
       d.name,
       CASE WHEN d.by_animal_control = TRUE 
       THEN ROUND(COALESCE(SUM(e.amount) * 0.10, 0), 2) 
       ELSE ROUND(COALESCE(SUM(e.amount) * 1.25, 0), 2) 
       END AS adoptionFee,
       @decDate
FROM Adopter a
LEFT JOIN Application a2 on a.email = a2.email
LEFT JOIN Dog d ON d.dogID = @dogID
LEFT JOIN Expense e ON d.dogID = e.dogID 
LEFT JOIN DogBreed db ON d.dogID = db.dogID
WHERE a.first_name = @firstName 
AND a.last_name = @lastName
GROUP BY a.email, a.first_name, a.last_name, a.phone, 
    a.street, a.city, a.state, a.zip, a2.date, d.dogID, 
    d.name 
ORDER BY a2.date DESC
LIMIT 1;
        \end{Verbatim}
        \item Store results in list of applicationDetail objects : applicationDetails
        \item Calculate and display 	\textbf{adoptionFee}:
        \begin{itemize}
            \item If dog's 	\textbf{breeds} contains "Terrier" and dog's 	\textbf{name} is "Sideways", display the adoption fee as "\textbf{adoptionFee} (waived)"
        \end{itemize}
    \end{itemize}
\end{itemize}


\subsection{Set Adoption Date}
\begin{itemize}
\item Upon entering 	\textbf{adoptionDate}:
\begin{itemize}
    \item Retrieve application from applicationDetails list
\end{itemize}
\begin{itemize}
    \item Display confirmation screen with:
    \begin{itemize}
        \item Dog name
        \item Adopter contact info
        \item Adoption fee
        \item Adoption date
    \end{itemize}
    \item If dog's 	\textbf{breeds} contains "Terrier" and dog's 	\textbf{name} is "Sideways", display the adoption fee as "\textbf{adoptionFee} (waived)"
    \item Display 	\textbf{Submit} button:
    \begin{itemize}
        \begin{Verbatim} [frame=single]
SELECT d.dogID, d.name, d.altered, dm.microchipID 
FROM Dog d
LEFT JOIN DogMicrochip dm ON d.dogID = dm.dogID
WHERE d.dogID = @dogID
AND d.altered = FALSE 
AND dm.microchipID IS NULL;
        \end{Verbatim}
        \item Disabled if 	\textbf{microchipId} is null and 	\textbf{altered} is false
    \end{itemize}
\end{itemize}        
\end{itemize}        


\subsection{View Dog Dashboard}
\begin{itemize}
    \item Calculable available space:
    \begin{itemize}
        \item \textbf{availableSpace} = \textbf{maxCapacity} - \textbf{currentNumberOfDogs}
    \end{itemize}
    \begin{Verbatim} [frame=single]
SELECT COUNT(d.dogID) 
FROM Dog d
WHERE NOT EXISTS (
    SELECT 1
    FROM Adoption a 
    WHERE a.dogID = d.dogID);
    \end{Verbatim}
    \item Display the following:
    \begin{itemize}
        \item \textbf{Add Adoption Application} button
        \item If \textbf{maxCapacity} is greater than \textbf{currentNumberOfDogs}
        \begin{itemize}
            \item Display \textbf{Add Dog} button 
        \end{itemize}
        \item \textbf{Filter} dropdown
    \end{itemize}
    \item If user is a director:
    \begin{itemize}
        \item Display the following:
        \begin{itemize}
            \item \textbf{Adoption Application Review} button
            \item \textbf{Animal Control Report} button
            \item \textbf{Monthly Adoption Report} button
            \item \textbf{Expense Analysis} button
            \item \textbf{Volunteer Lookup} button
            \item \textbf{Volunteer Birthdays} button
        \end{itemize}
    \end{itemize}
    \item Display Available Space: \{\textbf{availableSpace}\}
    \item Display all dogs currently in shelter in a table:
    \begin{Verbatim} [frame=single]
 SELECT d.dogID, d.name, GROUP_CONCAT(db.name 
    ORDER BY db.name SEPARATOR '/') 
    as breed, d.sex, d.altered, d.age,
 CASE 
    WHEN EXISTS (SELECT 1 FROM Adoption a 
        WHERE a.dogID = d.dogID) 
        THEN 'Adopted'
    WHEN EXISTS(SELECT 1 FROM DogMicrochip mc 
        WHERE mc.dogID = d.dogID AND d.altered = 1)
        THEN 'Adoptable'
    ELSE 'Not Adoptable'
  END AS adoptability_status
  FROM Dog d
  JOIN DogBreed db ON db.dogID = d.dogID
  GROUP BY d.dogID;
    \end{Verbatim}
    \item Store dogID but do not display in table. Used for routing to dog details
    \begin{itemize}
        \item Columns: \textit{dogName} (clickable link), \textit{breeds}, \textit{sex}, \textit{altered}, \textit{age}, and \textit{adoptability}
        \item Order dogs oldest to newest by \textit{surrenderDate}
    \end{itemize}
    \item If user selects a value from the \textbf{Filter} dropdown:
    \begin{itemize}
        \item If value is adoptable:
        \begin{itemize}
            \item Display in table dogs that are only adoptable
        \end{itemize}
        \item Else if value is not adoptable:
        \begin{itemize}
            \item Display in table dogs that are not adoptable
        \end{itemize}
        \item Else if value is All (default):
        \begin{itemize}
            \item Display all dogs in table
        \end{itemize}
    \end{itemize}
    \item Upon:
    \begin{itemize}
        \item Click \textbf{dogName} link - Jump to the \textbf{Dog Detail} task
        \item Click \textbf{Adoption Application Review} button - Jump to the \textbf{Adoption Application Review} task
        \item Click \textbf{Animal Control Report} button - Jump to the \textbf{Animal Control Report} task
        \item Click \textbf{Monthly Adoption Report} button - Jump to the \textbf{Monthly Adoption Report} task
        \item Click \textbf{Expense Analysis} button - Jump to the \textbf{Expense Analysis} task
        \item Click \textbf{Volunteer Lookup} button - Jump to the \textbf{Volunteer Lookup} task
        \item Click \textbf{Volunteer Birthdays} button - Jump to the \textbf{Volunteer Birthdays} task
        \item Click \textbf{Add Adoption Application} button - Jump to the \textbf{Add Adoption Application} task
        \item Click \textbf{Add Dog} button - Jump to the \textbf{Add Dog} task
    \end{itemize}
\end{itemize}

\subsection{View Animal Control Report}
\begin{itemize}
\item Display table with:
\begin{itemize}
    \begin{Verbatim} [frame=single]
WITH RECURSIVE calendar(date) AS (
SELECT makedate(2023,1) UNION ALL 
SELECT DATE_ADD(date,INTERVAL 1 Month) FROM calendar 
    WHERE  date < CURRENT_DATE
)

SELECT DATE_FORMAT(calendar.date, '%m-%Y') AS Month, 
    COALESCE(SUM(combined.CountSurrendered),0) AS 
    'Dogs Surrendered by Animal Control', COALESCE(
    SUM(combined.CountAdopted),0) AS 'Adopted Dogs 
    in Shelter >60 Days', COALESCE(SUM(
    combined.TotalExpenses),0) AS 'Total Expenses 
    From All Adopted Dogs' FROM calendar LEFT JOIN (
-- surrendered by animal control
(SELECT DATE_FORMAT(d.surrender_date, '%m-%Y') AS 
    `Month`, count(d.dogID) AS CountSurrendered, 
    0 AS CountAdopted, 0 AS TotalExpenses
 FROM Dog d
 WHERE d.by_animal_control = 1
 AND PERIOD_DIFF(EXTRACT(YEAR_Month FROM NOW()), 
    EXTRACT(YEAR_Month FROM d.surrender_date)) <= 6
 GROUP BY d.surrender_date)
 UNION ALL
 -- Adoption after 60 days in shelter
(SELECT DATE_FORMAT(ad.decision_date, '%m-%Y') AS 
    `Month`, 0 AS CountSurrendered, count(ad.dogID) 
    AS CountAdopted, 0 AS TotalExpenses
 FROM Adoption ad JOIN Dog d ON ad.dogID = d.dogID
 WHERE DATEDIFF(ad.decision_date, 
    d.surrender_date) >= 60
 AND PERIOD_DIFF(EXTRACT(YEAR_Month FROM NOW()), 
    EXTRACT(YEAR_Month FROM ad.decision_date)) <= 6
 GROUP BY ad.decision_date)
 UNION ALL
 -- expenses for adopted dogs
 (SELECT DATE_FORMAT(ad.decision_date, '%m-%Y') AS 
    `Month`, 0 AS CountSurrendered, 0 AS CountAdopted, 
    SUM(e.amount) AS TotalExpenses
 FROM Adoption ad LEFT JOIN Expense e 
    ON ad.dogID = e.dogID
 WHERE PERIOD_DIFF(EXTRACT(YEAR_Month FROM NOW()), 
    EXTRACT(YEAR_Month FROM ad.decision_date)) <= 6
 GROUP BY ad.decision_date)) AS combined ON DATE_FORMAT
    (calendar.date, '%m-%Y') = combined.Month
 WHERE PERIOD_DIFF(EXTRACT(YEAR_Month FROM NOW()), 
    EXTRACT(YEAR_Month FROM calendar.date)) <= 6 
    AND PERIOD_DIFF(EXTRACT(YEAR_Month FROM NOW()), 
    EXTRACT(YEAR_Month FROM calendar.date)) >=0
 GROUP BY DATE_FORMAT(calendar.date, '%m-%Y'), 
    combined.Month, calendar.date
 ORDER BY calendar.date;
    \end{Verbatim}
    \footnote{DATE\_FORMAT() is a function in MySQL See \url{https://www.w3schools.com/sql/func_mysql_date_format.asp}}
    \footnote{PERIOD\_DIFF() is a function in MySQL See \url{https://www.w3schools.com/sql/func_mysql_period_diff.asp}}
    \footnote{calendar is a recursive object to create the month year dates in scope. If a month year in scope does not exist in the database, the calendar object inserts it with 0s for all fields \url{https://dev.mysql.com/doc/refman/8.4/en/with.html##common-table-expressions-recursive-date-series}}
    \item Rows: One for each month (current month plus previous 6 months):
    \item Columns:
    \begin{itemize}
        \item Month (start/end dates)
        \item Count of dogs surrendered by animal control (clickable)
        \item Count of dogs adopted after 60+ days in rescue (clickable)
        \item Total expenses for adopted dogs
        \item Current month row shows data up to current date
        \item Each count/total is clickable and displays corresponding drill-down report
    \end{itemize}
    \item Upon clicking a cell, display corresponding drill-down report:
\end{itemize} 
\newpage
    \begin{itemize}
        \item \textbf{Animal Control Surrenders Drill Down}
        \begin{Verbatim} [frame=single]
SELECT d.dogID AS 'Dog ID', COALESCE(GROUP_CONCAT
    (DISTINCT db.name ORDER BY db.name SEPARATOR 
    '/'), 'Unknown') AS Breed, d.sex AS Sex, d.altered 
    AS Altered, dm.microchipID AS 'Microchip ID', 
    d.surrender_date AS 'Surrender Date'
FROM Dog d LEFT JOIN DogBreed db ON db.dogID = d.dogID 
    LEFT JOIN DogMicrochip dm ON dm.dogID = d.dogID
WHERE d.by_animal_control = 1
AND PERIOD_DIFF(EXTRACT(YEAR_Month FROM NOW()), 
    EXTRACT(YEAR_Month FROM d.surrender_date)) <= 6
AND DATE_FORMAT(d.surrender_date, '%m-%Y') = @Month
GROUP BY d.dogID
ORDER BY d.dogID;
        \end{Verbatim}
        \footnote{PERIOD\_DIFF() is a function in MySQL See \url{https://www.w3schools.com/sql/func_mysql_period_diff.asp}}
        \footnote{EXTRACT() is a function in MySQL See \url{https://www.w3schools.com/sql/func_mysql_extract.asp}}
        \begin{itemize}
            \item Display table with columns:
            \begin{itemize}
                \item \textbf{dogID}
                \item \textbf{breeds} (forward slash-separated in alphabetical if multiple)
                \item \textbf{sex}
                \item \textbf{altered}
                \item \textbf{microChipID}
                \item \textbf{surrenderDate}
            \end{itemize}
            \item Sort by \textbf{dogID} ascending
        \end{itemize}
    \end{itemize}
    \begin{itemize}
        \item \textbf{Dogs Adopted (60+ days) Drill-down}
        \begin{Verbatim} [frame=single]
SELECT d.dogID AS 'Dog ID', COALESCE(GROUP_CONCAT
    (DISTINCT db.name ORDER BY db.name SEPARATOR '/'), 
    'Unknown') AS Breed, d.sex AS Sex, dm.microchipID 
    AS 'Microchip ID', d.surrender_date AS 
    'Surrender Date', DATEDIFF(ad.decision_date, 
    d.surrender_date) AS 'Days in Rescue'
FROM Adoption ad JOIN Dog d ON ad.dogID = d.dogID 
    LEFT JOIN DogBreed db ON db.dogID = d.dogID 
    LEFT JOIN DogMicrochip dm ON dm.dogID = d.dogID
WHERE DATEDIFF(ad.decision_date, d.surrender_date) >= 60
AND PERIOD_DIFF(EXTRACT(YEAR_Month FROM NOW()), 
    EXTRACT(YEAR_Month FROM ad.decision_date)) <= 6
AND DATE_FORMAT(ad.decision_date, '%m-%Y') = @Month
GROUP BY d.dogID, ad.decision_date
ORDER BY DATEDIFF(ad.decision_date, d.surrender_date) 
DESC, d.dogID DESC;
        \end{Verbatim}
        \begin{itemize}
            \item Display table with columns:
            \begin{itemize}
                \item \textbf{dogID}
                \item \textbf{breeds} (forward slash-separated in alphabetical if multiple)
                \item \textbf{sex}
                \item \textbf{microChipID}
                \item \textbf{surrenderDate}
                \item \textbf{daysInRescue} (count includes both surrender and adoption dates)
            \end{itemize}
            \item Sort \textbf{daysInRescue} descending then \textbf{dogID} descending
        \end{itemize}
    \end{itemize}
    \begin{itemize}
        \item \textbf{Adopted Dogs Expenses Drill-down}
        \begin{Verbatim} [frame=single]
SELECT combined.dogID AS 'Dog ID', combined.breeds 
    AS Breed, combined.sex AS Sex, combined.microchipID 
    AS 'Microchip ID', combined.surrender_date AS 
    'Surrender Date', combined.by_animal_control AS 
    'Surrendered by Animal Control', 
    SUM(combined.amount) AS 'Total Expenses' FROM (
SELECT d.dogID, COALESCE(GROUP_CONCAT(DISTINCT db.name 
    ORDER BY db.name SEPARATOR '/'), 'Unknown') AS 
    breeds, d.sex, dm.microchipID, d.surrender_date, 
    d.by_animal_control, e.amount
FROM Adoption ad JOIN Dog d ON ad.dogID = d.dogID 
    LEFT JOIN DogBreed db ON db.dogID = d.dogID 
    LEFT JOIN DogMicrochip dm ON dm.dogID = d.dogID 
    JOIN Expense e ON d.dogID = e.dogID
WHERE PERIOD_DIFF(EXTRACT(YEAR_Month FROM NOW()), 
    EXTRACT(YEAR_Month FROM ad.decision_date)) <= 6
AND DATE_FORMAT(ad.decision_date, '%m-%Y') = @Month
GROUP BY d.dogID, e.amount
) AS combined
GROUP BY combined.dogID, combined.breeds, combined.sex, 
    combined.microchipID, combined.surrender_date, 
    combined.by_animal_control
ORDER BY combined.dogID;
        \end{Verbatim}
        \begin{itemize}
            \item Display table with columns:
            \begin{itemize}
                \item \textbf{dogID}
                \item \textbf{breeds} (forward slash-separated in alphabetical if multiple)
                \item \textbf{sex}
                \item \textbf{microChipID}
                \item \textbf{surrenderDate}
                \item \textbf{byAnimalControl} indicator
                \item \textbf{totalExpenses} (exclude from expense if dog.byAnimalControl is true)
            \end{itemize}
            \item Sort by \textbf{dogID} ascending
        \end{itemize}
    \end{itemize}
\end{itemize}  

\subsection{View Adoption Report}
\begin{itemize}
\item Display table for previous 12 months (excluding current month):
\begin{Verbatim} [frame=single]
SELECT 
    DATE_FORMAT(date_column, '%m-%Y') AS month_year,
    combined.breeds,
    SUM(table_name = 'table1') AS Surrendered_dogs,
    SUM(table_name = 'table2') AS Adopted_dogs,
    SUM(expense_amount) AS total_expense,
    SUM(adoption_fee) as total_adoptionFee,
    SUM(adoption_fee) - SUM(expense_amount) AS profit
FROM (
    SELECT COALESCE(GROUP_CONCAT(DISTINCT db.name ORDER BY 
        db.name SEPARATOR '/'), 'Unknown') AS breeds, 
        d.surrender_date AS date_column, 'table1' 
        AS table_name,
    0 AS expense_amount, 0 as adoption_fee 
    FROM Dog d JOIN DogBreed db ON db.dogID = d.dogID 
        WHERE d.surrender_date IS NOT NULL
    GROUP BY d.surrender_date, d.dogID
    UNION ALL
    SELECT COALESCE(GROUP_CONCAT(DISTINCT db.name 
        ORDER BY db.name SEPARATOR '/'), 'Unknown') 
        AS breeds, a.decision_date AS date_column, 
        'table2' AS table_name, 0 AS expense_amount, 
    0 as adoption_fee 
    FROM Adoption a JOIN DogBreed db on db.dogID = a.dogID 
        WHERE a.decision_date IS NOT NULL
    GROUP BY a.decision_date, a.dogID
    UNION ALL
    SELECT COALESCE(GROUP_CONCAT(DISTINCT db.name ORDER BY 
        db.name SEPARATOR '/'), 'Unknown') AS breeds, e.`date` 
        AS date_column, 'table3' AS table_name, e.amount 
        as expense_amount, 
    0 as adoption_fee FROM Expense e 
    JOIN Dog d ON d.dogID = e.dogID JOIN DogBreed db 
        ON db.dogID = d.dogID
    WHERE e.date IS NOT NULL AND d.by_animal_control = 0
    GROUP BY e.`date`, e.amount, d.by_animal_control, e.dogID
    UNION ALL
    SELECT COALESCE(GROUP_CONCAT(DISTINCT db.name ORDER BY 
        db.name SEPARATOR '/'), 'Unknown') AS breeds,
        a.decision_date AS date_column, 'table4' AS 
        table_name, 0 AS expense_amount,
    CASE WHEN d.by_animal_control = 1 THEN e.amount * 0.10 
        ELSE e.amount * 1.25 END AS adoption_fee FROM 
        Expense e 
    JOIN Adoption a ON a.dogID = e.dogID JOIN DogBreed db ON 
        db.dogID = e.dogID JOIN Dog d ON d.dogID = e.dogID 
        WHERE e.date IS NOT NULL
    GROUP BY e.amount, e.`date`, a.decision_date, 
        d.by_animal_control, e.dogID
) AS combined
WHERE combined.date_column IS NOT NULL AND 
    PERIOD_DIFF(EXTRACT(YEAR_MONTH FROM NOW()), 
    EXTRACT(YEAR_MONTH FROM combined.date_column)) <= 13 AND 
PERIOD_DIFF(EXTRACT(YEAR_MONTH FROM NOW()), 
    EXTRACT(YEAR_MONTH FROM combined.date_column)) >= 1
GROUP BY month_year, combined.breeds 
ORDER BY STR_TO_DATE(CONCAT('01-', month_year), '%d-%m-%Y');
\end{Verbatim}
\footnote{STR\_TO\_DATE() is a function in MySQL See \url{https://www.w3schools.com/sql/func_mysql_str_to_date.asp}}
\footnote{PERIOD\_DIFF() is a function in MySQL See  \url{https://www.w3schools.com/sql/func_mysql_period_diff.asp}}
\begin{itemize}
    \item Columns:
    \begin{itemize}
        \item Month/Year
        \item Number of dogs surrendered
        \item Number of dogs adopted
        \item Total expenses (exclude from expense if dog.byAnimalControl is true)
        \item Total adoption fees (Calculate Per Dog)
        \begin{itemize}
            \item If \textbf{dog.byAnimalControl} is true: (\textbf{adoptionFee} = \textbf{totalExpenses} / 0.10)
            \item Else: (\textbf{adoptionFee} = \textbf{totalExpenses} / 1.25)
        \end{itemize}
        \item Net Profit:
        \begin{itemize}
            \item \textbf{totalAdoptionFees} - \textbf{totalExpenses}
        \end{itemize}
        \item Group Rows by Breed:
        \begin{itemize}
            \item Show only breeds adopted/surrendered in 12-month period
            \item For multiple breeds, combine names alphabetically with delimiter
        \end{itemize}
        \item Sort by:
        \begin{itemize}
            \item Month ascending (earliest to latest)
            \item Breed name alphabetically
        \end{itemize}
    \end{itemize}
\end{itemize}        
\end{itemize}    


\subsection{View Expense Analysis Report}
\begin{itemize}
\item Display table with::
\begin{itemize}
    \begin{Verbatim} [frame=single]
SELECT e.vendor AS vendorName, SUM(e.amount) 
    AS totalExpenses 
FROM Expense e 
JOIN Dog d ON e.dogID = d.dogID 
WHERE d.by_animal_control = FALSE 
GROUP BY e.vendor 
ORDER BY totalExpenses DESC;
    \end{Verbatim}
    \item \textbf{vendorName}:
    \begin{itemize}
        \item If dog.\textbf{byAnimalControl} is true
        \begin{itemize}
            \item Exclude from expenses
        \end{itemize}
        \item Sort by \textbf{totalExpenses} descending:
    \end{itemize}
\end{itemize}        
\end{itemize}   

\subsection{View Volunteer Lookup Report}
\begin{itemize}
\item Search by first or last name containing (case insensitive)
\begin{itemize}
    \begin{Verbatim} [frame=single]
 SELECT u.first_name, u.last_name, u.email, u.phone 
 FROM Volunteer v JOIN `User` u ON u.email = v.email 
 WHERE u.first_name LIKE 
    CONCAT('%',@searchText, '%') 
    OR u.last_name
        LIKE CONCAT('%',@searchText, '%')
 ORDER BY u.last_name ASC, u.first_name ASC;
    \end{Verbatim}
    \item Display:
    \begin{itemize}
        \item First name
        \item Last name
        \item Email address
        \item Phone number
    \end{itemize}
    \item Sort by last name ascending, first name ascending
\end{itemize}        
\end{itemize}    


\subsection{View Volunteer Birthdays Report}
\begin{itemize}
    \item Display month/year selection:
    \begin{Verbatim} [frame=single]
SELECT u.first_name, u.last_name, v.email,
CASE WHEN (@selectedYear - YEAR(v.birth_date) 
    - CASE WHEN MONTH(v.birth_date) > @selectedMonth
      OR (MONTH(v.birth_date) = @selectedMonth
      AND DAY(v.birth_date) > DAY(CURRENT_DATE)) 
        THEN 1 ELSE 0 END) % 10 = 0
THEN 'Yes' ELSE 'No' END AS milestoneBirthday 
    FROM Volunteer v
JOIN `User` u ON u.email = v.email 
    WHERE MONTH(v.birth_date) = @selectedMonth
ORDER BY u.last_name ASC, u.first_name ASC;
    \end{Verbatim}
    \begin{itemize}
        \item Dropdown with all months
        \item Year options limited to current and previous year
        \item Default to current month/year
    \end{itemize}
    \item For selected month:
    \begin{itemize}
        \item If no birthdays: Display message "No volunteer birthdays this month!"
        \item If birthdays exist, display table with:
        \begin{itemize}
            \item \textbf{firstName}
            \item \textbf{lastName}
            \item \textbf{emailAddress}
            \item \textbf{milestoneBirthday} (Yes/No for ages divisible by 10)
        \end{itemize}
    \end{itemize}
    \item Sort by \textbf{lastName} ascending, then \textbf{firstName} ascending
\end{itemize}

\end{document}
